\documentclass[12pt, a4paper]{article}

\usepackage[utf8]{inputenc}
\usepackage[T1]{fontenc}
\usepackage[brazil]{babel}
\usepackage{geometry}
\usepackage{amsfonts, amsmath, amssymb}
\usepackage{graphicx}
\usepackage{hyperref}
\usepackage{titlesec}
\usepackage{indentfirst}
\usepackage{float}
\usepackage{caption}
\usepackage{animate}

\geometry{a4paper, margin={3cm,2cm,2cm,3cm}}
\setlength{\parindent}{1.25cm}

\captionsetup{
    font=footnotesize,
    labelfont=bf,
    position=top,
    justification=centering,
    singlelinecheck=false
}

% --- Configuração de Títulos (com titlesec) ---
% Força os títulos a terem 12pt (\normalsize) e negrito

\titleformat{\section}
  {\normalsize\bfseries\MakeUppercase} % formato: 12pt e negrito
  {\thesection.}         % rótulo (número): "1."
  {1em}                  % separação: 1 "M" de espaço
  {}                     % código antes do título (pode deixar vazio)

\titleformat{\subsection}
  {\normalsize\bfseries} % formato: 12pt e negrito
  {\thesubsection}      % rótulo (número): "1.1"
  {1em}                  % separação
  {}                     % código antes

\titleformat{\subsubsection}
  {\normalsize\bfseries} % formato: 12pt e negrito
  {\thesubsubsection.}   % rótulo (número): "1.1.1."
  {1em}                  % separação
  {}                     % código antes

\newcommand{\universidade}{Universidade Federal de Santa Catarina}
\newcommand{\centro}{Centro Tecnológico de Joinville - CTJ}
\newcommand{\disciplina}{Mecanismos}
\newcommand{\curso}{Engenharia Mecatrônica}
\newcommand{\professor}{Prof. Lucas Weihmann}
\newcommand{\titulo}{Projeto de Came máquina de costura}
\newcommand{\autori}{Gabriel A. Carvalho}
\newcommand{\autorii}{Gabriella A. Marques}
\newcommand{\autoriii}{Hebert A. Kubis}
\newcommand{\autoriv}{Jeiel S. A. Oliveira}
\newcommand{\autorv}{João V. F. Goetz}
\newcommand{\autorvi}{Johanna C. Rey}

\begin{document}
    
    \begin{titlepage}
        \begin{center}
            \begin{figure}[h!]
                \centering
                \includegraphics[width=.15\textwidth]{img/vertical_sigla_fundo_claro.png}
            \end{figure}
            {\bfseries \MakeUppercase{\universidade}}\\
            {\bfseries \MakeUppercase{\centro}}\\[3cm]

            {\MakeUppercase{\autori, \autorii, \autoriii\\
                            \autoriv, \autorv, \autorvi}}

            \vfill
            {\bfseries \MakeUppercase{\titulo}}\\
            \vfill

            {Joinville}\\
            {\number\year}

        \end{center}
    \end{titlepage}

    \tableofcontents
    \clearpage

    \section{Introdução}

    Este, tem como finalidade desenvolver o projeto de um came responsável pelo movimento vertical da agulha 
    em uma máquina de costura. Esse mecanismo realiza a conversão do movimento rotativo em deslocamento 
    linear periódico, sendo fundamental para garantir a regularidade e a qualidade do ponto produzido. 
    A precisão geométrica e cinemática do came influencia diretamente o desempenho do sistema, de modo que o 
    perfil adotado deve assegurar suavidade, repetibilidade e esforços reduzidos sobre o seguidor.

    Com base nos requisitos estabelecidos para o projeto, foi selecionado o perfil polinomial 3-4-5, 
    amplamente empregado em mecanismos de cames devido à continuidade de aceleração que proporciona e à 
    relativa simplicidade de modelagem. A partir desse perfil, foram determinadas as equações analíticas de 
    posição, velocidade e aceleração do seguidor ao longo de todo o intervalo de rotação de 360°. Em seguida, 
    foram elaborados os diagramas EVAP, o gráfico do ângulo de pressão e a análise dos valores máximos de 
    velocidade, aceleração e ângulo de pressão, parâmetros essenciais para a verificação da viabilidade 
    mecânica do mecanismo. O trabalho inclui, ainda, o desenho do came, a animação do conjunto came–seguidor 
    e o modelo para impressão tridimensional.

    \section{Requisitos de projeto}

        Os requisitos de projeto são os especificados:

        \begin{itemize}
            \item Raio de base: 19mm
            \item Raio do seguidor (rolete) = 6mm
            \item Altura máxima de deslocamnteo da agulha: 12mm
            \item Subida: 12mm em 50º
            \item Espera superior: em 10mm por 20º
            \item Descida: 12mm em 50º
            \item Repete-se essa sequência até completar 360º
            \item Velocidade de rotação: $\omega$ = 150 rpm
        \end{itemize}

    \section{Equacionamento}

        Para uma curva polinomial a expressão geral é dada por:

        \begin{equation}
            s = c_0 + c_1 \cdot \theta + c_2 \cdot \theta^2 + c_3 \cdot \theta^3 + ... + c_n \cdot \theta^n
        \end{equation}

        Onde:

        \begin{itemize}
            \item $\mathbf{s}$ = deslocamento do seguidor
            \item $\mathbf{\theta}$ = angulo de rotação do Came
            \item $\mathbf{c_i}$ = constantes (i = 1, 2, 3, ...)
            \item $\mathbf{n}$ - a ordem do polinomio
        \end{itemize}

        Como pedido no trabalho, o polinomio a ser utilizado deve ser um 3-4-5, ou seja:

        \begin{equation}
            s(\theta) = c_0 + c_1 \cdot \theta + c_2 \cdot \theta^2 + c_3 \cdot \theta^3 + c_4 \cdot \theta^4 + c_5 \cdot \theta^5
        \end{equation}

        \begin{equation}
            \frac{ds(\theta)}{d\theta} = c_1 + 2 \cdot c_2 \cdot \theta + 3 \cdot c_3 \cdot \theta^2 + 4 \cdot c_4 \cdot \theta^3 + 5 \cdot c_5 \cdot \theta^4
        \end{equation}

        \begin{equation}
            \frac{d^2s(\theta)}{d\theta^2} = 2 \cdot c_2 + 6 \cdot c_3 \cdot \theta + 12 \cdot c_4 \cdot \theta^2 + 20 \cdot c_5 \cdot \theta^3
        \end{equation}

        Ou na forma normalizada:

        \begin{equation}
            s(\theta) = h \cdot [10 \cdot (\frac{\theta - \theta_i}{\beta})^3 - 15 \cdot (\frac{\theta - \theta_i}{\beta})^4 + 6 \cdot (\frac{\theta - \theta_i}{\beta})^5]
        \end{equation}

        \begin{equation}
            \frac{ds(\theta)}{d\theta} = h \cdot 30 \cdot [(\frac{\theta - \theta_i}{\beta})^2 - 2 \cdot (\frac{\theta - \theta_i}{\beta})^3 + (\frac{\theta - \theta_i}{\beta})^4]
        \end{equation}

        \begin{equation}
            \frac{d^2s(\theta)}{d\theta^2} = h \cdot 60 \cdot [(\frac{\theta - \theta_i}{\beta}) - 3 \cdot (\frac{\theta - \theta_i}{\beta})^2 + 2 \cdot (\frac{\theta - \theta_i}{\beta})^3]
        \end{equation}
    
        \subsection{Posição}

            Para calcular encontrar as equações da posição, velocidade e aceleração, deve-se usar o parametro $\beta$ que é diferente para cada um 
            dos três trechos. O $\beta$ é definido como:

            \begin{equation}
                \beta = \theta_f - \theta_i
            \end{equation}

            Assim, tem-se:

            \begin{itemize}
                \item Trecho 1 (subida): $\beta$ = 50º - 0º = 50º = 0.873rad
                \item Trecho 2 (espera): $\beta$ = 70º - 50º = 20º = 0.349rad
                \item Trecho 3 (descida): $\beta$ = 120º - 70º = 50º = 0.873rad
            \end{itemize}

            Deste modo, a posição para o trecho 1 é dada por:

            \begin{equation}
                s(\theta) = 12 \cdot [10 \cdot (\frac{\theta}{0.873})^3 - 15 \cdot (\frac{\theta}{0.873})^4 + 6 \cdot (\frac{\theta}{0.873})^5]
            \end{equation}

            Para o trecho 2:

            \begin{equation}
                s(\theta) = 12
            \end{equation}

            E para o trecho 3, que possui uma pequena diferença por ser de descida, sendo que aqui a expressão anterior entra subtraindo de 1:

            \begin{equation}
                s(\theta) = 12 \cdot [1 - (10 \cdot (\frac{\theta - 1.222}{0.873})^3 - 15 \cdot (\frac{\theta - 1.222}{0.873})^4 + 6 \cdot (\frac{\theta - 1.222}{0.873})^5)]
            \end{equation}

            Com isto, o trecho de 0º a 120º pode ser representado. Ajustando os valores de $\theta_i$ tambem pode-se determinar as equações para o resto da rotação. 
            Para 120º < $\theta$ < 240º:

            \begin{equation}
                s(\theta) = 12 \cdot [10 \cdot (\frac{\theta - 2.094}{0.873})^3 - 15 \cdot (\frac{\theta - 2.094}{0.873})^4 + 6 \cdot (\frac{\theta - 2.094}{0.873})^5]
            \end{equation}

            \begin{equation}
                s(\theta) = 12
            \end{equation}

            \begin{equation}
                s(\theta) = 12 \cdot [1 - (10 \cdot (\frac{\theta - 3.316}{0.873})^3 - 15 \cdot (\frac{\theta - 3.316}{0.873})^4 + 6 \cdot (\frac{\theta - 3.316}{0.873})^5)]
            \end{equation}

            Para 240º < $\theta$ < 360º:

            \begin{equation}
                s(\theta) = 12 \cdot [10 \cdot (\frac{\theta - 4.189}{0.873})^3 - 15 \cdot (\frac{\theta - 4.189}{0.873})^4 + 6 \cdot (\frac{\theta - 4.189}{0.873})^5]
            \end{equation}

            \begin{equation}
                s(\theta) = 12
            \end{equation}

            \begin{equation}
                s(\theta) = 12 \cdot [1 - (10 \cdot (\frac{\theta - 5.411}{0.873})^3 - 15 \cdot (\frac{\theta - 5.411}{0.873})^4 + 6 \cdot (\frac{\theta - 5.411}{0.873})^5)]
            \end{equation}

            Assim, plotanto o grafico da posição:

            \begin{figure}[h!]
                \caption{Deslocamento do seguidor em função de $\theta$.}
                \vspace{0.5em}
                \centering

                \includegraphics[width=0.85\textwidth]{img/posicao.png}

                \vspace{0.5em}
                \centering
                \footnotesize\textit{Fonte: Dos Autores}
                \label{fig:posicao}
            \end{figure}
        
        \subsection{Velocidade}
            
            Para a velocidade os valores de $\beta$ continuam os mesmos. Assim, a equação da velocidade para o trecho 1 é:

            \begin{equation}
                v(\theta) = 360 \cdot \frac{15.708}{0.873} \cdot [(\frac{\theta}{0.873})^2 - 2 \cdot (\frac{\theta}{0.873})^3 + (\frac{\theta}{0.873})^4]
            \end{equation}

            Para o trecho 2:

            \begin{equation}
                v(\theta) = 0
            \end{equation}

            Para o trecho 3:

            \begin{equation}
                v(\theta) = - 360 \cdot \frac{15.708}{0.873} [(\frac{\theta - 1.222}{0.873})^2 - 2 \cdot (\frac{\theta - 1.222}{0.873})^3 + (\frac{\theta - 1.222}{0.873})^4]
            \end{equation}

            Para 120º < $\theta$ < 240º:

            \begin{equation}
                v(\theta) = 360 \cdot \frac{15.708}{0.873} \cdot [(\frac{\theta - 2.094}{0.873})^2 - 2 \cdot (\frac{\theta - 2.094}{0.873})^3 + (\frac{\theta - 2.094}{0.873})^4]
            \end{equation}

            \begin{equation}
                v(\theta) = 0
            \end{equation}

            \begin{equation}
                v(\theta) = - 360 \cdot \frac{15.708}{0.873} [(\frac{\theta - 3.316}{0.873})^2 - 2 \cdot (\frac{\theta - 3.316}{0.873})^3 + (\frac{\theta - 3.316}{0.873})^4]
            \end{equation}
            
            Para 240º < $\theta$ < 360º:

            \begin{equation}
                v(\theta) = 360 \cdot \frac{15.708}{0.873} \cdot [(\frac{\theta - 4.189}{0.873})^2 - 2 \cdot (\frac{\theta - 4.189}{0.873})^3 + (\frac{\theta - 4.189}{0.873})^4]
            \end{equation}

            \begin{equation}
                v(\theta) = 0
            \end{equation}

            \begin{equation}
                v(\theta) = - 360 \cdot \frac{15.708}{0.873} [(\frac{\theta - 5.411}{0.873})^2 - 2 \cdot (\frac{\theta - 5.411}{0.873})^3 + (\frac{\theta - 5.411}{0.873})^4]
            \end{equation}

            Assim, o gráfico da velocidade é:

            \begin{figure}[H]
                \caption{Velocidade do seguidor em função de $\theta$.}
                \vspace{0.5em}

                \centering
                \includegraphics[width=0.85\textwidth]{img/velocidade.png}

                \vspace{0.5em}
                \centering
                \footnotesize\textit{Fonte: Dos Autores}
                \label{fig:velocidade}
            \end{figure}

            Do gráfico da velocidade, fazendo os cálculos pelo geogebra, obtém-se que a velocidade máxima ocorre em 
            $\theta = 0.44 rad$ e tem valor de:

            \begin{equation}
                v(0.44) = 404.79 mm/s = 0.4047 m/s
            \end{equation}
        
        \subsection{Aceleração}
            
            Para a aceleração no trecho 1 tem-se:

            \begin{equation}
                a(\theta) = 720 \cdot (\frac{15.708}{0.873})^2 \cdot [(\frac{\theta}{0.873}) - 3 \cdot (\frac{\theta}{0.873})^2 + 2 \cdot (\frac{\theta}{0.873})^3]
            \end{equation}

            Para o trecho 2:

            \begin{equation}
                a(\theta) = 0
            \end{equation}

            Para o trecho 3:

            \begin{equation}
                a(\theta) = - 720 \cdot (\frac{15.708}{0.873})^2 [(\frac{\theta - 1.222}{0.873}) - 3 \cdot (\frac{\theta - 1.222}{0.873})^2 + 2 \cdot (\frac{\theta - 1.222}{0.873})^3]
            \end{equation}

            Para 120º < $\theta$ < 240º:

            \begin{equation}
                a(\theta) = 720 \cdot (\frac{15.708}{0.873})^2 \cdot [(\frac{\theta - 2.094}{0.873}) - 3 \cdot (\frac{\theta - 2.094}{0.873})^2 + 2 \cdot (\frac{\theta - 2.094}{0.873})^3]
            \end{equation}

            \begin{equation}
                a(\theta) = 0
            \end{equation}

            \begin{equation}
                a(\theta) = - 720 \cdot (\frac{15.708}{0.873})^2 [(\frac{\theta - 3.316}{0.873}) - 3 \cdot (\frac{\theta - 3.316}{0.873})^2 + 2 \cdot (\frac{\theta - 3.316}{0.873})^3]
            \end{equation}

            Para 240º < $\theta$ < 360º:

            \begin{equation}
                a(\theta) = 720 \cdot (\frac{15.708}{0.873})^2 \cdot [(\frac{\theta - 4.189}{0.873}) - 3 \cdot (\frac{\theta - 4.189}{0.873})^2 + 2 \cdot (\frac{\theta - 4.189}{0.873})^3]
            \end{equation}

            \begin{equation}
                a(\theta) = 0
            \end{equation}

            \begin{equation}
                a(\theta) = - 720 \cdot (\frac{15.708}{0.873})^2 [(\frac{\theta - 5.411}{0.873}) - 3 \cdot (\frac{\theta - 5.411}{0.873})^2 + 2 \cdot (\frac{\theta - 5.411}{0.873})^3]
            \end{equation}

            Assim, o gráfico da velocidade é:

            \begin{figure}[H]
                \caption{Aceleração do seguidor em função de $\theta$.}
                \vspace{0.5em}

                \centering
                \includegraphics[width=0.85\textwidth]{img/aceleracao.png}
                
                \vspace{0.5em}
                \centering
                \footnotesize\textit{Fonte: Dos Autores}
                \label{fig:aceleracao}
            \end{figure}

            Do mesmo modo que para a velocidade, pelo geogebra, obtem-se que a aceleração máxima ocorre em $\theta = 0.18 rad$ 
            tendo como valor máximo:

            \begin{equation}
                a(0.18) = 11209.76 mm/2^2 = 11.2098 m/s^2
            \end{equation}
        
    \section{ângulo de pressão}

        Para calcular o gráfico do ângulo de pressão, é necessário levar em conta o raio de base do came $R_b = 19 mm$, 
        a posição $S(\theta)$ do came e a velocidade $\frac{ds(\theta)}{d\theta}$. Com estes dados, o angulo de pressão é dado por:

        \begin{equation}
            \phi(\theta) = \arctan{\frac{\frac{ds(\theta)}{d\theta}}{R_b + s(\theta)}}
        \end{equation}

        Aplicando esta fórmula para cada trecho do deslocamento do came tem-se o seguinte gráfico do angulo de pressão: 

        \begin{figure}[H]
            \caption{Angulo de pressão em função de $\theta$.}
            \vspace{0.5em}

            \centering
            \includegraphics[width=0.85\textwidth]{img/angulo de pressão.png}
            
            \vspace{0.5em}
            \centering
            \footnotesize\textit{Fonte: Dos Autores}
            \label{fig:angulo de pressao}
        \end{figure}

        Sendo que no gráfico o ângulo está dado em radianos. Com o grafico plotado e usando a função intercect no geogebra 
        é possivel encontrar que o primeiro valor máximo de ângulo de pressão ocorre em $\theta = 0.39 rad = 22.35^\circ$ e que o valor deste 
        ângulo de pressão é:

        \begin{equation}
            \phi(0.39) = 0.75 rad = 42.73^\circ
        \end{equation}

        Com este valor pode-se verificar que é um valor ruim para utilização, pois as forças laterais no seguidor serão muito grandes 
        causando desgaste excessivo nas peças, portanto para que o projeto fosse viável seria necessário aumentar o raio de base para 
        um valor que tornasse $\phi_{max}$ menor que $30^\circ$.

        Analisando o primeiro trecho para um polinomio 4-5-6-7, tem-se que o a posição e a velocidade são dados por:

        \begin{equation}
            s(\theta) = 12 \cdot [15 \cdot (\frac{\theta}{0.873})^4 - 24 \cdot (\frac{\theta}{0.873})^5 + 10 \cdot (\frac{\theta}{0.873})^6]
        \end{equation}

        \begin{equation}
            v(\theta) = 720 \cdot  \cdot [(\frac{\theta}{0.873})^3 - 2 \cdot (\frac{\theta}{0.873})^4 + (\frac{\theta}{0.873})^5]
        \end{equation}

        Calculando o ângulo de pressão a partir desse trecho, derivando e igualando a 0 pode-se determinar que o ângulo de pressão 
        aumentou para $\approx45^\circ$. Logo, apesar de melhorar a questão dos solavancos no seguidor nas extremidades, o problema do 
        ângulo de pressão alto não se corrige somente aumentando o grau do polinomio, deve-se também mexer nas dimensões do came.

    \section{Modelagem e Simulação do Came}

        A partir das equações paramétricas obtidas para o perfil polinomial 3-4-5, procedeu-se à modelagem 
        tridimensional do came utilizando o software Autodesk Fusion 360, com o objetivo de validar a geometria 
        resultante e verificar a compatibilidade cinemática entre o came e o seguidor rolante. A modelagem 
        iniciou-se pela geração de um conjunto de pontos discretizados do contorno, obtidos diretamente das 
        funções de elevação $s(\theta)$, considerando o raio de base $R_b = 19 mm$ e o deslocamento radial total 
        do seguidor. Esses pontos foram importados como coordenadas polares e convertidos para coordenadas 
        cartesianas para construção da spline que define o perfil final.

        Após a reconstrução do contorno, realizou-se a extrusão do perfil para obtenção do sólido tridimensional 
        do came. O rolete seguidor, modelado com raio $r_f = 6 mm$, foi incorporado ao conjunto para permitir a 
        análise de contato. O acoplamento entre came e seguidor foi configurado por meio de restrições rígidas e 
        da definição de um joint rotacional no eixo do came, impondo a rotação contínua de $\omega = 150 rpm$.

        \begin{figure}[H]
            \caption{Modelagem do mecânismo no software Autodesk Fusion 360}
            \vspace{0.5em}

            \centering
            \includegraphics[width=0.85\textwidth]{img/modelagemCame.jpg}
            
            \vspace{0.5em}
            \centering
            \footnotesize\textit{Fonte: Dos Autores}
            \label{fig:modelagem mecanismo}
        \end{figure}

        Com o conjunto montado, foi executada a simulação cinemática direta, permitindo observar o comportamento 
        do seguidor para todo o ciclo de $360^\circ$. A análise evidenciou:

        \begin{itemize}
            \item Conformidade geométrica entre o perfil teórico e o movimento obtido pelo contato real com o 
            rolete;
            \item Coerência entre a elevação simulada e a curva analítica $s(\theta)$, confirmando a exatidão do 
            perfil importado;
            \item Identificação visual das regiões críticas onde a curvatura reduzida aumenta o risco de perda 
            de contato ou impacto;
            \item Correspondência entre a variação de velocidade aparente e os picos de velocidade obtidos 
            analiticamente, reforçando a consistência do modelo.
        \end{itemize}

        \begin{figure}[H]
            \caption{Simulação do mecânismo no software Autodesk Fusion 360}
            \vspace{0.5em}

            \centering
            \includegraphics[width=0.4\textwidth]{img/simulacao.png}
            
            \vspace{0.5em}
            \centering
            \footnotesize\textit{Fonte: Dos Autores}
            \label{fig:simulacao}
        \end{figure}


        Além disso, a simulação permitiu avaliar o comportamento do ângulo de pressão de forma qualitativa, 
        evidenciando o aumento do componente lateral da força transmitida ao seguidor nas regiões anteriormente 
        identificadas como críticas pela análise analítica. Tais observações reforçam a conclusão de que, apesar 
        da suavidade cinemática proporcionada pelo polinômio 3-4-5, o raio de base adotado é insuficiente para 
        limitar o ângulo de pressão a valores aceitáveis para operação contínua.

        Por fim, a modelagem permitiu a geração de um protótipo digital fiel, utilizado posteriormente para a 
        produção do modelo físico via impressão 3D, garantindo que o componente fabricado reproduz com precisão 
        o comportamento previsto pelas equações e pela simulação.

    \section{Conclusão}
    
        O desenvolvimento do came proposto permitiu obter um modelo completo a partir dos requisitos fornecidos 
        e do perfil polinomial 3-4-5. As equações de posição, velocidade e aceleração foram determinadas de 
        forma consistente e permitiram a construção dos diagramas EVAP, possibilitando uma análise detalhada 
        do comportamento cinemático do seguidor. A avaliação do ângulo de pressão revelou valores superiores a 
        40°, indicando esforços laterais elevados e sugerindo que ajustes geométricos, como o aumento do raio de 
        base, seriam necessários para aplicação prática.
        
        A análise também evidenciou que o aumento do grau do polinômio reduz os solavancos do movimento, embora 
        não seja suficiente para resolver o problema do ângulo de pressão isoladamente. O desenho técnico, a 
        animação do mecanismo e o modelo para impressão 3D complementaram o estudo, permitindo visualizar e 
        validar o funcionamento do came projetado. Assim, o trabalho atingiu os objetivos propostos, fornecendo 
        uma avaliação completa do perfil de came e destacando os fatores relevantes para seu emprego em sistemas 
        reais.

    \clearpage
    \begin{thebibliography}{9}

    \bibitem{Mecanismos}
    MARIA, José; BARBOSA, José M..
    \textit{Curvas de elevação}.
    Universidade Federal de Pernanbuco, 2022.
    Disponível em: \url{https://mecanismos.net.br/cames-introducao/}
    Acesso em: 09 de novembro de 2025.

    \bibitem{Mecanismos}
    MARIA, José; BARBOSA, José M..
    \textit{Ângulo de pressão}.
    Universidade Federal de Pernanbuco, 2022.
    Disponível em: \url{https://mecanismos.net.br/angulo-de-pressao/}
    Acesso em: 09 de novembro de 2025.

    \bibitem{design de mecanismos}
    ERDMAN, Arthur G..
    \textit{Mechanism design: analysis and synthesis}.
    4th ed. Upper Saddle River, NJ: Prentice Hall, 2001. v. ISBN 0130408727.

    \bibitem{elementos de maquinas}
    BUDYMAS, Richard G.; NISBETT, J. Keith.
    \textit{Elementos de máquinas de Shigley: projeto de engenharia mecânica}.
    8ª ed. Porto Alegre, RS: AMGH, 2011. 1084 p. ISBN 9788563308207.

    \bibitem{youtube}
    \textit{Perfil de came e funcionamento de máquina de costura}.
    Youtube, 2017. Disponível em: \url{https://www.youtube.com/watch?v=fZQDXpr_NTc}.
    Acesso em: 19 de novembro de 2025.

    \end{thebibliography}

\end{document}